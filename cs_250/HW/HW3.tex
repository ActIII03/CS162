% '%' is similar to '//' in C++ to mean a single line comment. The LaTeX compiler will ignore these lines.

\documentclass[11pt, a4paper]{article} 
\usepackage{amssymb}
\usepackage[fleqn]{amsmath} % fleqn: flush left equations <-- Left-flush equations
\usepackage{graphics}
\usepackage{hyperref, url}
\usepackage[margin=2cm, top=2cm]{geometry}
\newcommand\setItemNumber[1]{\setcounter{enumi}{\numexpr#1-1\relax}}
\newcommand*{\mybox}[1]{\framebox{#1}}
\newcommand{\tuple}[1]{$\langle #1 \rangle$} 
\newcommand{\stuple}[1]{$\langle$#1$\rangle$} % ordered pair with strings inside

\title{\bf Homework 1\\[1ex]
\rm\normalsize CS250 Discrete Structures I, Winter 2020 }
\date{\normalsize Due: April 12, 2020}
\author{\normalsize Armant Touche}

\begin{document} 
\vspace{0cm}\maketitle 
	\paragraph{Problem 1} Binary Relations\\
	These two exercises will require you to go outside the textbook to learn about binary relations. The TrevTutor video on relations is a good place to start: \url{https://www.youtube.com/watch?v=FI6j5QZNVx0}\\
	(Here I'm going to use the parentheses notation to mean angle brackets for ease of typing, feel free to also use this convenience, i.e., (a, b) is the ordered pair \stuple{a, b})
		\subparagraph{1.} Consider the following relations on the set $A = \{1, 2, 3\}$.
		\begin{enumerate}[a)]
			\item $R = \{(1, 1), (1, 2), (1, 3), (3, 3)\}$
			\item $S = \{(1, 1), (1, 2), (2, 1), (2, 2), (3, 3)\}$
			\item $T = \{(1, 1), (1, 2), (2, 2), (2, 3)\}$
			\item $\emptyset$ = empty relation
			\item $A \times A$ = universal relation on $A$
		\end{enumerate}
		Determine whether or not each of the relations is (i) reflexive, (ii) symmetric, (iii) transitive, and (iv) antisymmetric.
		
		\subparagraph{2.} Consider the following relations:
		
		\begin{enumerate}[a)]
			\item Relation $\leq$ (less than or equal) on the set $\mathbb{Z}$ of integers.
			\item Set inclusion $\subseteq$ on a collection $C$ of sets.
			\item Relation $\perp$ (perpendicular) on the set $L$ of lines in the plane.
			\item Relation $\parallel$ (parallel) on the set $L$ of lines in the plane.
			\item Relation $\mid$ of divisibility on the set $\mathbb{N}$ of positive integers. ($x \mid y$ if there exists $z$ such that $xz = y$.)
		\end{enumerate}
		
		Determine whether or not each of these relations is (i) reflexive, (ii) symmetric, and (iii) transitive.	

	\paragraph{Problem 2} Textbook Chapter 0.4 Functions Exercises (page 51-56)\\
	
	Complete exercises 3, 4, 6, 8, 19, 28
		
\end{document}
