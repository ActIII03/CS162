\documentclass[11pt,a4paper]{article} 
\usepackage{amssymb}
\usepackage[shortlabels]{enumitem}
\usepackage[fleqn]{amsmath} 
\usepackage{relsize}
\usepackage{graphicx}
\usepackage[super]{nth}
\usepackage{hyperref, url}
\setcounter{secnumdepth}{0} 
\usepackage[margin=2cm, top=2cm]{geometry}
\graphicspath{{./img/}}
\newcommand\setItemNumber[1]{\setcounter{enumi}{\numexpr#1-1\relax}}
\newcommand*{\mybox}[1]{\framebox{#1}}
\newcommand{\tuple}[1]{$\langle #1 \rangle$} 
\newcommand{\stuple}[1]{$\langle$#1$\rangle$} % ordered pair with strings inside
\newcommand{\forceindent}{\leavevmode{\parindent=1em\indent}}
\DeclareUnicodeCharacter{2212}{-}
\title{\bf Homework 9\\[1ex]
\rm\normalsize CS250 Discrete Structures I, Winter 2020 }
\date{\normalsize Due: June, 3, 2020}
\author{\normalsize Armant Touche}

\begin{document} 
\vspace{0cm}\maketitle 

\section{Powerset}
    
        \begin{enumerate}

        \item \textit{Proof}:

        In my program, the set of bit strings of length $n$ (variable: power\_set\_size) is used to expressed $\{0, 1\}^n$. This list is used as postional reference for each element of the original set $A$ in which we generating the power set for. The positional reference is a yes (1) or no (0) reference to the set of subset ($\mathcal{P}(A)$). To create a relationship between the bit string and $A$ we need to create a bijective relationship.

        We first define $f: \{0, 1\}^n\rightarrow \mathcal{P}(A)$ as $f((a_1, a_2,...,a_n)) = \{i|a_i = 1\}$. What this function means is that bit strings map to the set of all positions 1's in the string. For example, n = 4, bit string is (1, 0, 1, 1), this would map to to $\{1, 3, 4\}$. To highlight bijectivity, we need to prove injection and surjection

        \item Injective:

        Suppose that $x$ and $y$ both map to one subset $S\in \mathcal{P}(A)$. $S$ defines the yes (1) or no (0) postional reference to this single subset. If $f(x) = f(y) = S$, $x$ and $y$ must have 1' in exactly the same position for the subset to exist. $n$, $x$, and $y$ are identical because bit string must only contain 0's and 1's. Therfore $f$ is injective because at the most 1 bit string can map to a single subset

        \item Surjective

        Examine a subset $S\in \mathcal{P}(A)$. Again, $S$ only contains integers between 1 and $n$, inclusively speaking. The list $x = (x_1, x_2,...,x_n)$ can be made such that all for all $i\in S$, $x_i = $, and all others are 0. $f$ state that list $x$ will map exactly to $S$. Thus, since all subsets $S$ are mapped to by some list $x$ representing a bbbit string of length $n$, $f$ is surjective.

        With these two existing relationship of mappings, a bijective relationship exist between the set of bit strings of length $n$ and the $\mathcal{P}(A)$ so thes 2 sets have equal cardinality.

        \end{enuemrate}



            
        
        

\end{document}
