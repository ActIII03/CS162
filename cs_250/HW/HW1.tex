% catalogue of LaTex pkg: https://www.biu.ac.il/os_site/documentation/teTeX/help/Catalogue/index.html

\documentclass[11pt, a4paper]{article} 
%\usepackage{amssymb,amsthm,listings,color,array,easytable,multicol} 
\usepackage[fleqn]{amsmath} % fleqn: flush left equations <-- Left-flush equations
\usepackage{graphics}
\usepackage{hyperref, url}
\usepackage[margin=2cm, top=2cm]{geometry}
\newcommand{\tuple}[1]{$\langle #1 \rangle$} 
\newcommand\setItemNumber[1]{\setcounter{enumi}{\numexpr#1-1\relax}}

\title{\bf Homework 1\\[1ex]
\rm\normalsize CS250 Discrete Structures I, Winter 2020 }
\date{\normalsize Due: April 5, 2020}
\author{\normalsize Armant Touche}

\begin{document} 

\vspace{0cm}\maketitle 

\paragraph{Problem 1} Figure out how to typeset the following mathematical statements.

\begin{enumerate}

    \item $f(x) =  \mathcal{O}(n \log n)$ (Big `O' notation)

    \item $\neg(A \land B) \leftrightarrow (\neg A \lor \neg B)$ (De Morgan's law in propositional logic notation)

    \item $\overline{A \cup B} = \overline{A} \cap \overline{B}$ (De Morgan's law in set theory notation)

    \item $f(x) = \log_{2} x^{2}$ (Subscripts and superscripts)

    \item $A = \frac{\pi d^2}{4}$ (Fraction and special symbols)

    \item $S = \{a, b, c, d\}$ (A set definition)

    \item (Truth Table)
        \begin{align*} 
			\begin{array}{|c c|c|} 
			p & q & p \land q\\ 
			\hline 			
            T & T & T\\
			T & F & F\\
			F & T & F\\
			F & F & F\\
			\end{array}
        \end{align*}
    \item (A summation statement)
		\begin{align*}
			\sum_{k=1}^{n}n 
		\end{align*}

\end{enumerate}

\paragraph{Problem 2} Read chapter 0.1 and 0.2 of the textbook and write up solutions to the following exercises (page 17--23 in the pdf version).

%(Hint: All of these have solutions or a check your answers field in the online version of the textbook: \url{http://discrete.openmathbooks.org/dmoi3/sec_intro-statements.html})

\begin{enumerate}
    % (1)
    \item For each sentence below, deicide it is an atomic statement, a molecular statment, or not a statement at all.
        \begin{enumerate}
            \item Customers must wear shoes
            \item The customers wore shoes
            \item The customers wore shoes and they wore socks
        \end{enumerate}
    
    % (3)
    \setItemNumber{3}
    \item Supposse $P$ and $Q$ are the statements: $P$: Jack passed math. $Q$: Jill passed math.
        \begin{enumerate}
            \item Translate "Jack and Jill both passed math" into symbols
            \item Translate "If Jack passed math, then Jill did not" into symbols
            \item Translate "$P \lor Q$" into English
            \item Translate "$ \neg(P \land Q) \rightarrow Q$" into English
            \item Suppose you know that if Jack passed math, then so did Jill. What can you conclude if you know that:
                \begin{enumerate}
                    \item Jill passed math?
                    \item JIll did not pass math?
                \end{enumerate}
        \end{enumerate}

    % (10)
    \setItemNumber{10}
    \item Write each of the following statements in the form, “if . . . , then . . . .” Careful, some of the statements might be false (which is alright for the purposes of this question).
        \begin{enumerate}
            \item To lose weight, you must exercise.
            \item To lose weight, all you need to do is exercise.
            \item Every American is patriotic.
            \item You are patriotic only if you are American.
            \item The set of rational numbers is a subset of the real numbers
            \item A number is prime if it is not even.
            \item Either the Broncos will win the Super Bowl, or they won’t play in the Super Bowl.
        \end{enumerate}

    % (12)
    \setItemNumber{12}
    \item Let $P(x)$ be the predicate, "$3x + 1$ is even"
        \begin{enumerate}
            \item is $P(5)$ true or false?
            \item What, if anything, can you conclude about $\exists xP(x)$ from the truth value of $P(5)$?
            \item What, if anything, can you conclude about $\forall xP(x)$ from the truth value of $P(5)$?
        \end{enumerate}

    % (16)
    \setItemNumber{16}
    \item Translate into symbols. Use $E(x)$ for " $x$ is even" and $O(x)$ for "$x$ is odd."
        \begin{enumerate}
            \item No number is both even and odd.
            \item One more than any even number is an odd number.
            \item There is prime number that is even.
            \item Between any two numbers there is a third number.
            \item There is no number between a number and one more than that number.
        \end{enumerate}

    % (17)
    \setItemNumber{17}
    \item Translate into English:
        \begin{enumerate}
            \item $\forall x (E(x) \rightarrow E(x + 2)$
            \item $\forall x \exists y(\sin(x) = y)$
            \item $\forall y \exists x(\sin(x) = y)$
            \item $\forall x \forall y(x^3 = y^3 \rightarrow x = y)$
        \end{enumerate}
\end{enumerate}
\end{document}
