% catalogue of LaTex pkg: https://www.biu.ac.il/os_site/documentation/teTeX/help/Catalogue/index.html

\documentclass[11pt, a4paper]{article} 
%\usepackage{amssymb,amsthm,listings,color,array,easytable,multicol} 
\usepackage[fleqn]{amsmath} % fleqn: flush left equations <-- Left-flush equations
\usepackage{graphics}
\usepackage{hyperref, url}
\usepackage[margin=2cm, top=2cm]{geometry}
\newcommand{\tuple}[1]{$\langle #1 \rangle$} 

\title{\bf Homework 1\\[1ex]
\rm\normalsize CS250 Discrete Structures I, Winter 2020 }
\date{\normalsize Due: April 5, 2020}
\author{\normalsize Armant Touche}

\begin{document} 

\vspace{-2cm}\maketitle 

\paragraph{Problem 1} Figure out how to typeset the following mathematical statements.

\begin{enumerate}

    \item $f(x) =  \mathcal{O}(n \log n)$ (Big `O' notation)

    \item $\neg(A \land B) \leftrightarrow (\neg A \lor \neg B)$ (De Morgan's law in propositional logic notation)

    \item $\overline{A \cup B} = \overline{A} \cap \overline{B}$ (De Morgan's law in set theory notation)

    \item $f(x) = \log_{2} x^{2}$ (Subscripts and superscripts)

    \item $A = \frac{\pi d^2}{4}$ (Fraction and special symbols)

    \item $S = \{a, b, c, d\}$ (A set definition)

    \item (Truth Table)
        \begin{\hspace{1mm}align*} 

			\begin{array}{|c c|c|} 
			p & q & p \land q\\ 
			\hline % Put a horizontal line between the table header and the rest.
			T & T & T\\
			T & F & F\\
			F & T & F\\
			F & F & F\\
			\end{array}
        \end{align*}
    \item (A summation statement)
		\begin{align*}
			\sum_{k=1}^{n}n 
		\end{align*}
\end{enumerate}

\paragraph{Problem 2} Read chapter 0.1 and 0.2 of the textbook and write up solutions to the following exercises (page 17--23 in the pdf version)\\

Exercises: 1, 3, 10, 12, 16, and 17.\\
\end{document}
