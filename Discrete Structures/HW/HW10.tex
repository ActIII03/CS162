\documentclass{letter}
\usepackage[super]{nth}
\usepackage{hyperref}
\signature{Armant Touche}
\begin{document}

\begin{letter}{Name: Armant Touche \\ Class: CS 250 
\\ }
\opening{To David Lu,}

Again, Appreciate your openness to feedback and allowing us to self-evaluate.\\\\
1. Did you set yourself realistic goals for your term? Did you meet them?\\\\
I set out to study consistently each day for time management reasons. I felt I achieved about 80\% of my study goals which kind of makes me feel bad. I didn't let that failure stress me out because I kept reminding myself that college is more about learning how to learn and when I feel like I didn't achieve a short-term goal, I remind myself that college isn't a race, it's marathon.\\\\
2. Reflecting back to the start of this term, did you learn much?\\\\
I felt I learned a decent amount given the current circumstances. I really like the binary relations and set theory concepts because I feel that is very transferable, real-world skill.\\\\
3. What were some interesting things you learned?\\\\
Again, I like the binary relation and set theory concepts that were introduced to us. I also greatly appreciated the chance to use \LaTeX\ and learning how to typeset in Vim with the vimtex plugin. The more I progress through college, I am beginning to see a somewhat-present, scaffolding-structure called Computer Science skills that I posses, metaphorically speaking. For instance, learning how to write proofs is perhaps very useful and persuasive if used correctly. Being able to communicate a concept without errors is crucial skill to posses/\\\\
4. Of the subjects you’ve covered, where do you feel you still have gaps in your knowledge/skill?\\\\
The  binomial coefficient concepts is probably my weakest concept.\\\\
5. How did you utilize your time studying? Are there ways you could make that more efficient?\\\\
Some days, I didn't consistently study for one reason or another but I felt like I was at least consistent in delivering my assignments.\\\\
6. Did you learn anything about your own mode of learning? For instance, do you find that you learn better or worse by watching videos? Or were there problem solving strategies that you were able to use more or less effectively?\\\\
I can adapt to synchronous or asynchronous learning but I prefer asynchronous to be honest mixed with group activities that call for collaboration. Working in isolation can be not as fulfilling to learning a concept versus learning in a group. Of course, do reading the beforehand is crucial to carrying out asynchronous learning because concept aren't read by osmosis.\\\\
7.  Did you revisit earlier work that you completed, even if it was correct and work it through again? (Can be a powerful reinforcement learning technique.)\\\\
In the beginning, I did but until the \nth{5} week, I began to not revisit because other classes mixed with personal situations started to consume my time.\\\\
8. How did the outside circumstances of your life affect your learning this term?\\\\
Towards the \nth{5} week, I was not bothered with any extenuating circumstances. This last week though was weird because of the protest and wanting to contribute something positive versus being inactive and quiet. We are living in extraordinary times and I feel paradigm shift is needed. I do not have anything specifically to address but I think interdependence and humans interact needs reexamination. Thanks for reading this David, you're very understanding.


\closing{Take care,}

\end{letter}
\end{document}
