% '%' is similar to '//' in C++ to mean a single line comment. The LaTeX compiler will ignore these lines.

\documentclass[11pt]{article} % a back slash '\' denotes a command in LaTeX. The documentclass defines the standard layout of the document. Square brackets '[]' are parameters that can be passed to the argument.
\usepackage{amssymb,amsthm,listings,color,array,easytable,multicol} % packages like these bring in additional features via commands. They're like #include in C++
\usepackage[fleqn]{amsmath} % fleqn: flush left equations
\usepackage{venndiagram} % Here are a bunch more packages that I use frequently. Google them to find out what they are used for.
\usepackage{graphicx}
\usepackage{hyperref, url}
\usepackage[margin=2cm, top=2cm]{geometry}
\usepackage[shortlabels]{enumitem}
\newcommand{\tuple}[1]{$\langle#1\rangle$} % The \newcommand command is like a macro in C++
\newcommand{\stuple}[1]{$\langle$#1$\rangle$} % ordered pair with strings inside

%The title, date, and author are part of the preamble. The preamble is not part of the main body of the document. 
\title{\bf Homework 4\\[1ex]
\rm\normalsize CS250 Discrete Structures I, Winter 2020 }
\date{\normalsize Due: April 26, 2020}
\author{\normalsize David Lu}

\begin{document} % This \begin{document} and corresponding \end{document} identifies the begeinning and end of your document body.

	\vspace{-4cm}\maketitle % \maketitle tells LaTeX to put the title, date, and author information here.
	
	\begin{center}
		\fbox{\fbox{\parbox{5.5in}{\centering
					Your solutions must be typed (preferably typeset in \LaTeX{}) and submitted as a PDF on D2L.
					\\
					
					Solutions generally require clear explanations in complete sentences and must be in your own writing.
		}}}
	\end{center}
	
	\paragraph{Assignment For Week4}
	
	\begin{itemize}
		\item Read chapter 1.1, 1.2, 1.3, and 1.4 in the textbook on functions. \href{http://discrete.openmathbooks.org/dmoi3.html}{http://discrete.openmathbooks.org/dmoi3.html}
		\item Watch the TrevTutor videos on counting, combinations, and permutations. (Videos 26-36 from this playlist are relevant: \url{https://www.youtube.com/playlist?list=PLDDGPdw7e6Ag1EIznZ-m-qXu4XX3A0cIz})
		\item Complete the following exercises.
	\end{itemize}

	\paragraph{Counting, Combinations and Permutations} Chapter 1.1, 1.2, 1.3, and 1.4\\
	
	This week, I'm asking you to cover a bit more ground than in previous weeks. That's because most of us are already somewhat familiar with the topic. This week, we're going to learn how to count. I'm not joking. \\
	
	However, it turns out that counting is actually harder than it sounds. For example, consider the following questions:
	
	\begin{itemize}
		\item How many 4 digit numbers do not contain any repeated digits?
		\item How many possible distinct games of Chess are there?
		\item How many hands beat a 3 of a kind of Kings in Poker?
		\item If $|S| = 9$, what is $\{|X : X \in \mathcal{P}(S), |X| \leq 4\}|$?
	\end{itemize}

	Some of these questions seem really hard at first (and some just are really hard). But there are some tools and tricks to help us. First, we will think of counting in terms of sets. \\
	
	For instance:
	\begin{itemize}
		\item How big is the set of 4 digit numbers with no repeated digits?
		\item How big is the set of distinct games of Chess?
		\item How big is the set of poker hands that beat 3 kings?
	\end{itemize}

	This may seem like a small change from the first set of questions, but now we're talking about sets and we can bring to bear all the concepts, operations, and tools from set theory to help solve the problem.\\
	
	\textbf{The fundamental theorem of counting}, stated in set theory is as follows:
	if $|A| = m$ and $|B| = n$ then $|A \times B| = m \cdot n$. In other words, if the cardinality of set $A$ is $m$ and the cardinality of set $B$ is $n$, then the cardinality of the Cartesian product $A \times B$ is $m$ times $n$.\\ 
	
	In short, if we combine two tasks, the numbers multiply. If there are $m$ distinct ways of doing one thing and $n$ distinct ways of doing another, then there are $m \cdot n$ ways to do both things. In the textbook, this is also called the \textbf{Multiplicative Principle} or the \textbf{Rule of Product}.
	
	

	
	\paragraph{Exercises} Chapter 1: Counting\\
	
	I'll try to keep most of the homework exercises from the textbook for consistency. I recommend working through problems beyond what's assigned for the homework to more thoroughly evaluate your own understanding of the material. If you have time, do all the exercises. Many of them have solutions and check yourself hints in the online version of the textbook.
	
	\paragraph{Problem 1} 1.1 Additive and Multiplicative Principles (pages 67-69)\\
	
	From section 1.1 in the textbook, complete exercises 4 and 10
	
	\paragraph{Problem 2} 1.2 Binomial Coefficients  (pages 78 - 80)\\
	
	From section 1.2 in the textbook, complete exercises 3, 4, and 11
	
	\paragraph{Problem 3} 1.3 Combinations and Permutations (pages 86 - 88)\\
	
	From section 1.3 in the textbook, complete exercises 1, 5, 6, 12
	
	\paragraph{Problem 4} Counting Proofs\\
	
	Given what you've learned from chapter 1.4 about the difference between calculation and proofs, write up a proof or explanation for the following statement:\\
	
	\textbf{A full binary tree of depth $d$ has $2^{d-1}$ leaves.} \\
	
	(Recall that a full binary tree is a binary tree where every path is the same length, i.e. all the leaf nodes are at the same depth.)
	
		
\end{document}