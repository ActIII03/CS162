\documentclass[12pt,a4paper,man,natbib]{apa6}

\usepackage[english]{babel}
\usepackage[utf8x]{inputenc}
\usepackage{amsmath}
\usepackage{natbib}
\usepackage{graphicx}

\title{Inquiry Logs}
\shorttitle{Inquiry Logs}
\author{Armant Touche}
\affiliation{Portland State University}
\date{\today}

\abstract{}

\begin{document}
\maketitle

\subsection{Quotes}

Directions: Pick quotes from the readings and media. Include at least THREE QUOTES PER LOG, and not more than FIVE, ideally from multiple sources. At least one should be from the Kimmerer book, others can be from the assigned material and optional additional material.  Write the quotes directly (could be just a few words and typically not more than 30-40) and include an in-text citation for the quote that includes author, year, and original page number (or time in the case of video quotes).  All supplemental readings will have enough information to make a complete in-text citation.

\subsection{Observe}

Directions:Pick a place to sit or walk once a week that is OUTSIDE; a covered patio/porch is OK.  Spend about 15-30 minutes at this activity and ideally sit or walk AFTER completing some of the readings for the week, and BEFORE you write your log.  While sitting/walking, observe the environment around you.  Use your senses: listen, smell, look, touch, and taste. Now, contemplate the readings. When you return to the log, write briefly about this experience. What did you notice, what were you thinking about related to the course readings, assignments, or topics?  Did this time focus your thoughts, or expand them, or perhaps you were distracted by other thoughts? What was it like to spend time thinking and observing in this way?  Aim for 150-200 words.

\subsection{Reflect \& Theme}

Directions: Write a reflection relating the quotes and text to your own experience, interests, and/or concerns. (Do not repeat the quotes already listed, but refer to them, e.g. "my first quote selection" or "the Kimmerer quote".) Reflect on how/why these particular pieces of text speak to you and how it helps you inquire and learn about the natural sciences. After completing the entire readings (and other sources) for the week, identify a theme you found that connected across authors.  What in the readings made you identify this theme?  Include additional quotes with in-text citations if it helps you explain the theme, but this is not expected. This piece is your own reflection.  Bold and underline your theme so it stands out and is clear.  Aim for 400-600 words.


\subsection{Question}

Directions:  Identify one (or two, but not several) open-ended question(s), anchored in the readings, that the readings raised for you and that you feel is relevant to the class.  Pose the question and then reflect on it. As you do, remember that you play an integral role in the inquiry. Your answer will influence the class. It should not be a rhetorical question, nor should it be intended to “test” your peer’s knowledge. Rather, pose a question that genuinely means something to you – a question that you think would be interesting to pursue with your peers.  Aim for 200-300 words.

\subsection{Works Cited}

Directions:Write an APA reference list for materials cited in this log.  If you use a citation generation program, check for errors in the program.  If you prefer to use MLA or Chicago, that is OK too but be consistent in the style that you use.  The goals of citations in a works cited list is to be clear, complete and consistent.  All supplemental readings will have enough information to make a complete Work-Cited list.

\subsection{Words}

Directions: List the total number of words not counting the quotes and works cited.  The minimum length is 700 words of original text; the maximum length is 1,200 words. Ideally, the entire log fits on 2 or at most 3 typed pages \cite{BobsBook01}


\bibliography{mybib}
\bibliography{apalike}

\end{document}
