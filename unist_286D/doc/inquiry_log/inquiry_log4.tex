\documentclass[a4paper,man,biblatex]{apa6}
\usepackage[american]{babel}
\usepackage{csquotes}
\usepackage[backend=biber]{biblatex}
\usepackage[document]{ragged2e}
\setlength{\RaggedRightParindent}{0.5in}
\addbibresource{../references/references.bib}
\usepackage{url}
\usepackage{xpatch}
\xpatchbibdriver{online}
  {\printfield{entrysubtype}}
  {\printfield{entrysubtype}%
   \newunit\newblock
   \printfield{note}}
  {}
  {}

\renewcommand{\abstract}[1]{}

\title{Inquiry Log \#3}
\shorttitle{Inquiry Log \#4}
\author{Armant Touche}
\affiliation{Portland State University}
\date{\today}

\begin{document}
\thispagestyle{otherpage}
\setcounter{biburllcpenalty}{7000}
\setcounter{biburlucpenalty}{8000}

%\maketitle

\noindent Name: Armant Touche\newline
\noindent Date: 5/20/2020

\subsection{Quotes}
%#1
\begin{enumerate}
    \item "Biodiversity starts in the distant past and it points toward the future." (Frans Lanting)
    \item "Emergency situations can lead to basic control systems being suspended or bypassed" \autocite{account_pan}
    \item "A stable planet is a prerequisite to have good, human well-being" \autocite{stable} 

\end{enumerate}

\subsection{Observe} For this observation period I choose my backyard, again. Mostly this week, it has been raining which is nice for our plants in the backyard. Even with the disruption, plant life still thrive with or without us (humans). Some of my quotes reflect my feeling towards this concept which is numbing and full of feeling at the same time. Thankfully for my situation, I am able to dwell more so on the positive feeling more than the numbness that occasionally accompanies reality. I used to grow Portobello mushrooms, and I was always fascinated by mycelium and how fungi reproduce. As I am taking a breath, there is no doubt in my mind that I am breathing in some traceable amount of spores. The chaos and complexity behind mycelium growth is complex in each individual stage of growth, yet is simple when observing from a grower's perspective. The three main factors that contribute to fungi growth are humidity, warmth, and varying amounts of light, depending on what is the current stage of mycelium growth. For the grower, I just have to make those conditions come true. The first stage is inoculation which requires very sterile conditions because as I mentioned earlier, there a plethora of spores in the air and I wouldn't want to grow mushrooms or other fungi because in some cases, the spores can spawn very hazardous fungi or even mushrooms. The other stages of growth require different input variables but I will go into detail in this log. Anyway, I was mostly thinking about the how plants and other organisms grow without humans. 

\subsection{Reflect \& Theme} For the theme for this inquiry log, the theme is biodiversity. I had a quote from an earlier inquiry log that disparaged the monoculture present in most parts of the agriculture industry. I was thinking about that while watching \textit{Dirt Rich} and \textcite{dirtrich} also mentioned that the average amounts of crops that will be harvested in the future averages around twenty-six harvest left because of the lack of diversity in our agricultural industry. If that is true, not only current generations will experience issues with crops due to monoculture but our children will too. A possible solution to this perhaps citizens can start growing some of their own sustenance. Still, parts of the population will not be able to afford that. Not being able to help or do their own farming seems to be the case for a lot of people and especially college students. Not only did my generation experience the '08 Recession but now, we are experiencing this current pandemic. Money is the mostly readily used tool in influencing progressive initiatives and my generation is at a greater disadvantage than most generations growing up in America. Still, with what money our generation has, we can choose to not support organizations that choose to not change with current circumstances. Current circumstances dictate that human could use some more introspection in regard to how we operate in the world and how we interact with one another. 

\subsection{Question} My question for this inquiry log is, how can governments better triage issues that need our (humans) immediate attention? Well, since I am technologists, may be we can benefit to more transparency from our governments. There should be more initiatives on centralizing data and standardizing reporting systems in a way that is digestible to citizens. I am suggesting a reporting system similar to open source websites like Github but for scientists. The term scientist can be used in a broader sense in a way that even a fourteen year-old can perhaps contribute their input. Using the open source approach to boasting transparency can be a factor that makes topic like climate change or biodiversity more accessible to the layman and not silo'd. I'm using silo in terms of sociology because I feel specialized-terminology and bureaucracy get in the way of innovation. There would fail-safes  in place to prevent manipulation in a away to change narratives for special agendas that do not benefit everyone. This solution to better aide governments in prioritizing issues may be idealistic and would require implementation of a solid infrastructure. Governments already have issues maintaining already implemented systems. Take for instance during the shutdown in early 2019. U.S. government, during the longest shutdown, weren't renewing critical Secure Sockets Layer certificates (SSL) which are super important to ensuring secure protocols are in place \autocite{ssl_govern}. Of course, the shutdown wasn't the cause but the timing was horrible and possibly resulted in a lot of data being exposed to the Internet and citizen's personal identifying information may have been leaked. All in all, there should be triage systems in place that allow people from any walk of life to be able to contribute to helping solve current and future issues.

\subsection{Words} 804 words 

\printbibliography

\end{document}
