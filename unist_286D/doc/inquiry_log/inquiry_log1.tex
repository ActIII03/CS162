\documentclass[a4paper,man,biblatex]{apa6}
\usepackage[american]{babel}
\usepackage{csquotes}
\usepackage[backend=biber]{biblatex}
\addbibresource{references.bib}

\renewcommand{\abstract}[1]{}

\title{Inquiry Logs}
\shorttitle{Inquiry Logs}
\author{Armant Touche}
\affiliation{Portland State University}
\date{\today}

\begin{document}

%\maketitle

\section{Quotes}
%#1
\begin{enumerate}
    \item "The potentials for plants were seen purely through the lens of animal capacity."\autocite[19]{braiding}
    \item "Why can't we compare and contrast with another way of doing science from an indegenious perspective" @ 3.48 \autocite{twoeyes}
    \item "This method [Native Science], also an organized belief system, has sustained Native peoples and cultures for millennia against nearly overwhelming odds.  But, because of this world view, traditional peoples often find themselves ill-prepared to protect their own best interest." \autocite{westvsnative}
\end{enumerate}

\section{Observe}

        I set out on walk from house Thursday morning-ish with the intent to stretch my legs since being cooped up in my house for eighteen days but whose counting. As I began to distance myself from my house, a slight mist occurred as did a overcast too. I have like-dislike relationship with the weather but I try to remain indifferent to the weather here because that is normal for the PNW. My goal for the walk was to walk at least five miles and listen to no music. The feel of the moisture is gradually becoming more saturated on my face but I do not mind. I noticed pedastrians on the various points on the sidewalk hurrying to escape this mist. I ask why? The faster you move through the rain or mist, the more you become saturated. I always remembered this when I used to do ruck-runs in the military when it used to run. Just enjoy it. So I enjoyed it. Of course, it doesn't hurt to have an umbralla. I had one. Reading \textit{Brading Sweetgrass} and this walk reminds me of the dichotomy between man and nature. Humans are part of the animal kingdom, which comprises not just of humans avoiding the rain but the squirrels traveling from tree to tree or the birds flying while it is raining.  

\section{Reflect \& Theme}
I never took the time to ponder the shortcomings of Western science but I think the reading \textit{Brading Sweetgrass} begs the question, "Why don’t you care to know more about  native science?" The answer is simple which is, "Native science or any of it’s teachings was never a subject in my school." Pretty straightforward for me and probably a large majority of American public-school attendees. I felt catapulted into public school with no questions asked. I try to maintain a curiosity about nature because growing up in the South, I was very much an outdoors-person and you could always catch me in the woods or looking to climb tree. I loved to climb trees, especially Magnolia Trees because they were shorter compared to the common pine-tree. I was expose to the death very early because my father would take me deer hunting. Before going on the hunt, him and I would prepare our gear and practice shooting targets. Target practice, if carried out right, would carry over to a humane death for the soon-to-be, stalked deer. Now that maybe a little graphic but growing up poor and the farthermost grocery being two hours ways, we couldn’t and definitely didn’t waste any part of the deer. The lack of wasting Nature’s materials emphasizes the same philosophy Kimmerer described when see was harvesting sap to make syrup. Also, white-tail were and still are considered the destructive to crops that farmers work hard to supply to the community. There are counter-points to that argument but growing up, that was what I was told. Deer served a purpose in for many Indigenous tribes \autocite{nativedeer}. Deer also served a purpose for family growing up so practicing not being wasteful was ingrained in me from an early age and think Kimmerer was ingrained with similar teachings too.\newline 
\parThere is balance with everything around us and continuously maintaining a good reciprocity relationship between nature and man is crucial to sustaining a healthy environment. Another theme is liked was Kimmererr imaginative approach to relating animacy to her studies. Often, I feel I sometimes lack the her level appreciation towards validating animals and inanimate ojects as living beings. Even rocks get the same level of validation which is unusual but I understand the sentiments she posses and I do not think she has an over exaggerated attitude towards animals and inanimate objects. Overall, I feel very secure with my openess and I think we benefit with having Native Science being a subject taught in public schools. 

\section{Question}

Do you think we (students) can benefit from using the compare and contrast method when comparing Western Science and Native Science? I think students from any background can benefit from examining the difference between Western Science viewpoint of it's surroundings and Indeginuos People's viewpoint of it's surroundings. I feel that so-called "disordered" systems often contain novel emergent properties that can be beneficial. For instance, when Kimmerer was collecting sap and after collection, she used cold temperatures to create syrup versus the modern way of burning wood to heat the sap into a syrup. She noticed the parameters required to making a syrup and tweak the syrup making process in a way that was more environmentally kosher, so to speak. There maybe other novel approaches like Kimmerer's way of making syrup that may also be a solution to modern problem.  

\section{Words}

Directions: List the total number of words not counting the quotes and works cited.  The minimum length is 700 words of original text; the maximum length is 1,200 words. Ideally, the entire log fits on 2 or at most 3 typed pages

\printbibliography

\end{document}
