\documentclass[a4paper,man,biblatex]{apa7}
\usepackage[american]{babel}
\usepackage{siunitx}
\usepackage[backend=biber]{biblatex}
\usepackage{float}
\usepackage{capt-of}
\usepackage[document]{ragged2e}
\setlength{\RaggedRightParindent}{0.5in}
\addbibresource{../references/references.bib}
\usepackage{graphicx}
\graphicspath{{./img/}}
\usepackage{url}
\usepackage{xpatch}
\usepackage{siunitx}
\let\paragraph\oldparagraph
\let\subparagraph\oldsubparagraph
\usepackage{titlesec, blindtext, color}

%Formatting
\titlespacing{\section}
{0pt}{-1ex}{-1ex}

\title{Streamflow Variability}
\shorttitle{Streamflow Variability}
\author{Armant Touche}
\affiliation{Portland State University}
\date{\today}

\begin{document}
%\maketitle
\noindent Name: Armant Touche\newline
\noindent Class: UNST-286D\newline
\noindent Instructor: Dr. Bernadette Rodgers\newline
\section{Introduction} 
\par In this research paper, I will be describing how researchers measure disturbance in regard to streamflow variability, the reasoning behind  using the "disturbance index", and current cases of variability having an affect in the Southwest region of the United States. Streamflow refers to the amount of water flowing in a river at any given time \autocite{streamflow_def} in a hydrological system. The "disturbance index" is measurement tool used by the United States Environmental Protection Agency (USEPA) to measure the amount of disturbance present in watershed regions across the nation \autocite{falcone_2016}. The reason for needing the index is to use this measurement, along with other variables, is to track the behavior of these hydrological systems. Changes in hydrological systems, especially streamflow, in the Continental United States (CONUS) have caused some negative outcomes where one being an increase in floods in some regions \autocite{rice_2016}. A regional example of a negative outcome from increased streamflow variability is the increasing number of floods in California \autocite{standford_2020}. The scale of variability is still being studied since previous methods used datasets that contained redundancies, which decreased accuracy but \textcite{falcone_2016} described the best approach to accurately classifying watershed disturbance. There is a relationship between streamflow variability and both water resource availability and management \autocite{rice_2016}. For present and future studies in relation to hydrological systems, understanding the effects on water resources cause by streamflow variability is crucial to measure human activity and the impact of the side effects from such activity in order to positively manage water resources here in the CONUS and elsewhere in the world.\\
\section{Data Analysis}  
\par To better understand the negative outcomes from increased streamflow variability and increasingly changing climate, the \textcite{mallakpour_2018} study reported streamflow minimum and maximum annual streamflows mean (\si{\cubic\meter\per\second}) comparing watershed in Northern California.\\ 
\par In \textit{Figure 1}, notice the minimum streamflow values (red-down arrows) in panels (D-F). In D-panel, there are more reports on negative trends occurring in streamflow values from 1950-2005 which means that inland streams are displaying a decreasing mean during the dry seasons and in G-panel, during the wet season, there is increase in the mean streamflow value. The increase in mean during the wet-season promotes more floods and decrease in mean during dry-season promotes more droughts \autocite{mallakpour_2018}. I will not go into detail on each variable's effect because USGS Although the trends are minor from 1950-2005, the cumulative distribution for future trends highlight a negative trend. Cumulative distribution is likelihood or probability for a non-discrete value to occur at any given point ranging from one to zero, usually referred to as a random variable. In the current context, the random variable is streamflow discharge (\si{\cubic\meter\per\second}) \autocite{cdf_def}.\\
%Fig. 1
 \begin{minipage}{0.65\linewidth}   
     \includegraphics[scale=0.35]{stream_flow_cali.png}
     \captionof{figure}{Future Streamflow trends in California}
     \label{fig:streamflow_trend}
\end{minipage}
\vspace{2ex}
\par \textit{Figure 2} highlights cumulative trend which displays the cumulative distribution of three dataset -- (1) 1950-2005 (2) RCP 4.5 2020-2099 (3) RCP 8.5 2020-2099. The left panel is Oroville Lake in California and the right panel is Shasta Lake. Notice the blue line's distribution is greater than the two project trends from RCP 4.5 and RCP 8.5. RCP 4.5 is a model of future streamflow using using a bias-corrected projection and RCP 8.5 uses data derived from global climate model (GCM) to simulate future streamflows. GCM data is known to have systematic error (biases) due to limited GIS spatial resolution which is why one (RCP 4.5) of the two future projections uses a bias correction projection to account for theses errors (biases).\\
%Fig. 2
\begin{minipage}{0.65\linewidth}   
    \centering
    \includegraphics[scale=0.40]{cdf_streamflow.png}
     \captionof{figure}{Empirical Cumulative Distribution Functions (ECDFs) of streamflow in the baseline (blue line) and projection periods (red line RCP 4.5 and green line RCP 8.5)}
\end{minipage}
\vspace{2ex}
\par But how does climate change affect streamflow variability? From \textcite{rice_2016} study, atmospheric variables may be potential drivers in streamflow variability. \textcite{rice_2016} results suggest that human activities may magnify or amplify the expression of changes. $P_\textit{mean}$ (Precipitation) and $DI_\textit{mean}$ (Dryness index) which were the important atmospheric variables that acted as drivers. There were more variables being used by United State Geological Survey (USGS) but watershed climatology, for instance was too varied amongst CONUS watersheds. \textit{Figure 3} models streamflow trends across the CONUS \autocite{rice_2016}. The red markers depict a decreasing trend in streamflow mean values and blue depicts an increase in streamflow mean values. Again, drawing our notice to the Southwest region, the increased number cases of floods in California highlights the relationship with the blue markers which model increased streamflow variability. \\
%Fig. 3
 \begin{minipage}{0.65\linewidth}   
     \centering
    \includegraphics[scale=.4]{conus_trend}
     \captionof{figure}{The estimated magnitude of trends in monthly streamflow from 1940 to 2009. }
\end{minipage}
\section{Results}
\par To understand how human activity contributes to streamflow variability within streams, let's discuss the human "disturbance index" mentioned \textcite{falcone_2016} study. Building roads, damning streams, and inputs from agricultural sources and urban sources serve as contributors to changes in stream ecosystems. \textcite{falcone_2016} was not the only study to try to measuring human activity. In \textcite{stein_2002}, they mentioned that previous studies used biological markers as a way of measuring human activity (i.e. soil sample and other biological harvesting methods) which can be time consuming and are site-specific. The sample from one site does not necessarily mean that it represent the whole. Still, the duty of monitoring human effects of environment should be a priority when it comes to urban and landscape planning \autocite{stein_2002}. With the advent of the Geographic Information System (GIS), better data can be derived which uses multiple variables versus the single variable approach described in \textcite{stein_2002} and other studies \autocite{falcone_2016}. The biggest take away from \textcite{falcone_2016} is that a reduced variable set can reproduce inferences from different watershed sites and the main six variables that should be used to measure human's impact on streams are: (1) housing-unit density (housing unit \si{\per\square\kilo\meter}), (2) Road density in watershed (\si{\kilo\meter\per\square\kilo\meter}), (3) Sum of 43 major pesticide compounds (\si{\kilo\gram\per\square\kilo\meter}), (4) Urban-crops-pasture land cover in a 600 m mainstem buffer, (\%), (5) Average linear distance of sampling site to all canals/ditches (m), (6) Dam storage in basin (liters $\times$ 1000 \si{\square\kilo\meter}). Although the focus can be centered around these variables, there is still an issue with deriving GIS data on small-to-medium sized watershed. The reason for difficultly is that the resolution is sometimes too low for accurate spatial analysis hence why the variables are not to reliant on variables involving spatial quantities. Examples of spatial quantities are most of the variables from the National Land Cover Data-set \autocite{falcone_2016}. Even with resolution errors, the results from this study suggest that "disturbance index" is an objective, deductionist approach to measuring humans activities' effects on streamflow and the data analysis conducted by \textcite{mallakpour_2018} suggests that anthropogenic effects are affecting streamflow  and will continue effecting hydrologically systems.\\
\section{Discussion} Water and climate change are quickly becoming global and political issues and many of these issues relate with national security and human health. Before the 70's, obtaining topological images without the aide of computers was a chore that many scientists did not like to do. Even after receiving images or maps, the accuracy may not or may have been the best because of the decentralized nature of information and standards back then. With better systems for acquiring maps, like the Internet, has boosted research interests into hydrological systems in order to understand water and climate change issues \autocite{bras_1999}. In \textcite{bras_1999} article, he believes the studies of groundwater hydrology increased scientists' interest into studying the impact of certain variables that affect certain properties with hydrology, particularly flow and transport.\\
\section{Conclusion} The results from \textcite{rice_2016} conclude that climate change does and has affected streamflow in CONUS which serve as sources for freshwater and using indexes like \textcite{falcone_2016} can objectively quantify the impact from human activity like urban developing and water flow controls. It's no wonder that CONUS is experiencing more droughts in some areas like the Pacific Northwest and more floods in areas like Southern California. The usage of GIS-derived data can be prone to errors low pixel resolution but still, the "disturbance index" is a useful \textit{a priori} method for measuring anthropogenic effects on streamflow. Future streamflow models predict increasing variability and that cause difficult for water resource management. In order to combat this difficult task, urban and landscape planners need to understand that certain variables need better consideration in regards to housing, water controls, and other variables mentioned in this research paper. \\
\printbibliography
\end{document}
