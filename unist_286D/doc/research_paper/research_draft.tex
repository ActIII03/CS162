\documentclass[a4paper,man,biblatex]{apa6}
\usepackage[american]{babel}
\usepackage{csquotes}
\usepackage[backend=biber]{biblatex}
\usepackage[document]{ragged2e}
\setlength{\RaggedRightParindent}{0.5in}
\addbibresource{references.bib}
\usepackage{url}
\usepackage{xpatch}
\xpatchbibdriver{online}
  {\printfield{entrysubtype}}
  {\printfield{entrysubtype}%
   \newunit\newblock
   \printfield{note}}
  {}
  {}

\renewcommand{\abstract}[1]{}

\title{Inquiry Log \#3}
\shorttitle{Inquiry Log \#3}
\author{Armant Touche}
\affiliation{Portland State University}
\date{\today}

\begin{document}
\thispagestyle{otherpage}
\setcounter{biburllcpenalty}{7000}
\setcounter{biburlucpenalty}{8000}

%\maketitle

\noindent Name: Armant Touche\newline
\noindent Date: 4/29/2020

\subsection{Summary} 

\subsection{Draft Thesis} The non-existence of a central or common application interface for citizens to access fails to underpin the Environmental Science community's dataset worth and does not instill awareness in the Information Technology (IT) community who are needed to develop such applications \autocite{Viqueira_2020} I want to share some known challenges in implementing IT.


\printbibliography

\end{document}
