\documentclass[a4paper,man,biblatex]{apa6}
\usepackage[american]{babel}
\usepackage{csquotes}
\usepackage{siunitx}
\usepackage[backend=biber]{biblatex}
\usepackage[document]{ragged2e}
\setlength{\RaggedRightParindent}{0.5in}
\addbibresource{references.bib}
\usepackage{url}
\usepackage{xpatch}
\xpatchbibdriver{online}
  {\printfield{entrysubtype}}
  {\printfield{entrysubtype}%
   \newunit\newblock
   \printfield{note}}
  {}
  {}

\renewcommand{\abstract}[1]{}

\title{Research Summary \& Draft Thesis}
\shorttitle{Research Summary \& Draft Thesis}
\author{Armant Touche}
\affiliation{Portland State University}
\date{\today}

\begin{document}
\thispagestyle{otherpage}
\setcounter{biburllcpenalty}{7000}
\setcounter{biburlucpenalty}{8000}

%\maketitle

\noindent Name: Armant Touche\newline
\noindent Date: 4/29/2020
\section{Streamflow Variability}
\subsection{Summary} There is research on streamflows variability and how human activity which is between 40-60\%, respectively, using linear and nonlinear statistical models \autocite{streamflow}. These models are geared towards modeling time series variables like rain-fall run-off, base-flow, and a couple other hydrological variables. The interest of this research topic is to reassess and emphasize already known predictions. Also highlight the climate change variables commonly used in this area of research that contribute to streamflow variability. Reason I selected this topic because I want to learn more about human contribution to climate change specifically in regard to changes in river flow. The interest stemmed from being around bodies of water most of my life, I cannot imagine the time it took for these bodies of waters to form. Even more to my main point, I would like to learn about how humans contribute to these changes in bodies of water and what are some key contributing factors to streamflow variability. The article I shared in conjunction with this summary and my draft thesis mostly describes streamflow changes in China near the Heihe River Basin, but I am using the methodology present in the shared as a loose framework as I research more about stream flow prediction here in the United States.

\subsection{Draft Thesis} The purpose of my research is to prove that humans' activities contribute to streamflow change and to quantify how human much contribute (i.e. man-made water structure, agriculture, and other yet to be stated). Data that will be used will mostly stem from the public domain. Possible variables that will be used are spring streamflow, temperature, and average daily discharge \SI{}{\meter\cubed\per\second}. 

\printbibliography

\end{document}
