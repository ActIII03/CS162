\documentclass{article}
\usepackage{hyperref}
\usepackage[margin=1in]{geometry}
\usepackage{graphicx}
\setcounter{secnumdepth}{0} 
\graphicspath{{./img/}}

\begin{document}
\title{\vspace{-2cm}Presentation Outline}
\date{June 3, 2020}
\author{Armant Touche}
\maketitle

\section{Outline}

\begin{enumerate}
    
    \item Talk about what is streamflow:

    \begin{itemize}

        \item (Define Streamflow) The U.S. Geological Survey (USGS) uses the term “streamflow” to refer to the amount of water flowing in a river.
        \item (Highlight the importance of rivers) Rivers are invaluable to not only people, but to life everywhere. Not only are rivers a great place for people (and their dogs) to play, but people use river water for drinking-water supplies and irrigation water, to produce electricity, to flush away wastes (hopefully, but not always, treated wastes), to transport merchandise, and to obtain food. Rivers are major aquatic landscapes for all manners of plants and animals. 

    \end{itemize}

    \item Talk about the how the amount of water changes:

    \begin{itemize}

    \item (Define Streamflow Variability) Discharge rate of streams around the world are are changing as a result from dams, land-use changes, water diversions and changing climate patterns.

    \item (Talk about Mallakpour) To highlight current streamflow changes, one data analysis performed by Mallakpour et al. (2018) measured changes within inland watersheds in California and highlighted significant changes that occurred between 1950 and 2005. Panel (D), on the second row, are trends negative trends during the dry seasons and the red arrows suggest that more streams aren't discharge less each year. Panel (G), on the third row, suggests that streams during the wet seasons are discharging more over time which is why during El Nino's, the are more floods occurring in Southern California each year. The second and third column are future projections where each use a different approach to predicting future trends from the same data derived from global climate model. Uses data from the Global Climate Model dataset, across the U.S., Mallakpour et. al (2018) were also abale to project future trends and the variability are highlighted by the figure. The red markers signify a decreasing trend which means regions with red markers are expected to see less streamflow. Less streamflow means a higher probability of droughts and such. Regions with blue markers are expected to see an increase. Again, places like Southern California are expected to higher streamflow discharges thus more floods during the wet season. It's known that we are seeing earlier transitions from spring and vice versa for transitions into winter. Same goees for streamflow discharges, we are seeing earlier discharges from streams and later decreases in flow when transitioning into winter.

    \end{itemize}

    \item Highlight the relationship between human activity and streamflow

        \begin{itemize}

            \item (Talk about Falcon 2016) First, how to quantify human activity without bias and errors was the mean focus of a research study done by Falcone et al. (2016) set out to come up a objective, deductionist index or measurement for human activity. Deduction is opposite of reduction in which the deduction take data from population(s) versus sample(s) and tries to infer something meaningful. In the study, Falcone et. al (2016) stated that old ways of measuring human's impact on streamflow usually involved site sampling which can cause data from certain to not match or haver a relationship with others (reduction). A better to way understand and measure human activity, the variables listed on the slide were the most valid variables to use when measuring human activity's impact on streamflow variability. I will not go into detail about each variable due to the complexity of certain ones like Sum of 43 major pesticide compounds because the USGS already using them.

        \end{itemize}

    \item Streamflow Research

        \begin{itemize}

            \item Water and climate change are quickly becoming global and political issues. Before the 70's, obtaining topological images without the aide of computer was chore no researcher wanted to do. Even after receiving images or maps, the accuracy may or may not have been the best due to the decentralized nature of informations and standard back then. With the advent of GIS, more hydrological research is being had. There are still issues with spatial images being too pixilated on small-to-medium watershed with objective measurements like the human "disturbance index," more studies be able to be reproduce in other region where research on streamflow variability in particular are being carried out. 

        \end{itemize}
\end{enumerate}

\end{document}
