\documentclass{article}
\usepackage{hyperref}
\usepackage[margin=1in]{geometry}
\usepackage{graphicx}
\graphicspath{{./img/}}

\begin{document}
\title{\vspace{-2cm}Presentation Outline}
\author{Armant Touche}
\maketitle

\section{Outline}

\begin{enumerate}
    
    \item Talk about what is streamflow:

    \begin{itemize}

        \item (Define Streamflow) The U.S. Geological Survey (USGS) uses the term “streamflow” to refer to the amount of water flowing in a river.
        \item (Highlight the importance of rivers) Rivers are invaluable to not only people, but to life everywhere. Not only are rivers a great place for people (and their dogs) to play, but people use river water for drinking-water supplies and irrigation water, to produce electricity, to flush away wastes (hopefully, but not always, treated wastes), to transport merchandise, and to obtain food. Rivers are major aquatic landscapes for all manners of plants and animals. 

    \end{itemize}

    \item Talk about the how the amount of water changes:

    \begin{itemize}

    \item (Define Streamflow Variability) Discharge rate of streams around the world are are changing as a result from dams, land-use changes, water diversions and changing climate patterns.

    \item (Talk about Mallakpour) To highlight current streamflow changes, one data analysis performed by Mallakpout et al. (2018) measured changes within inland watersheds in California and highlighted significant changes that occurred between 1950 and 2005. Panel (D), on the second row, are trends negative trends during the dry seasons and the red arrows suggest that more streams aren't discharge less each year. Panel (G), on the third row, suggests that streams during the wet seasons are discharging more over time which is why during El Nino's, the are more floods occurring in Southern California each year. The second and third column are future projections where each use a different approach to predicting future trends from the same data derived from global climate model. 

    \end{itemize}

    \item Highlight the relationship between 

\end{enumerate}

\end{document}
