\documentclass{article}
\usepackage{hyperref}
\usepackage[margin=1in]{geometry}

\begin{document}
\title{\vspace{-2cm}Data on Sources}
\author{Armant Touche}
\maketitle

\section{Guidelines}

    \subsection{Keyword Searches}
    \begin{enumerate}

        \item Search engine used:
        \item Keywords:
        \item How many results:
        \item How many were peer-reviewed:
        \item How many are from 2017 or newer:
        \item Calculate \% new peer-reviewed:

    \end{enumerate}


    \subsection{Augment Keyword Searches (Use synonyms)}
    \begin{enumerate}

        \item Search engine used:
        \item Keywords:
        \item How many results:
        \item How many were peer-reviewed:
        \item How many are from 2017 or newer:
        \item Calculate \% new peer-reviewed:

    \end{enumerate}

    \subsection{Other Metrics}
    \begin{enumerate}

        \item Number of abstracts read:
        \item How many articles read: 
        \item Calculate \% of articles read out number of abstracts read:

    \end{enumerate}

\section{List of sources}

\begin{enumerate}

    %1
    \item Rice - \textit{Influence of watershed}...\\ Notes: (I) Abstract: Analyzed 70 years of annual scale streamflow (1940-2009) in CONUS. Focus of study was streamflow variability and at the CONUS scale, the balance between precipitation and evaporative demand, and measures of geographic location were important. Boosted regression tree model was used to study the relationship between watershed characteristics and the magnitude of trends. (II) Intro: A decline in streamflow has been reported in analyses to specific regions. Existing research focused heavily on discrete time intervals but the research is lacking widespread, long-term changes in the periodic structure of streamflow time series. The questions asked are: (1) What patterns emerge in changes in the magnitude of annual scale streamflow variability across the CONUS between 1940 and 2009? (2) How are characteristics of individual water sheds related to variation in the magnitude of those trends? (3) How do these relationships vary spatially? (III) Methods: (Data Overview) Datasets consists of 967 watersheds within the CONUS chosen from the USGS GAGES-II reference dataset which are considered minimally impacted by human activity. (Wavelet transform and streamflow) Continuous Wavelet transform (CWT) was applied to quantify the magnitude of annuals scale variability. Instead of discrete WT, the CWT provides a more effective extraction of information from geophysical time series. (Spatial data) Analysis of relationship between individual watersheds and the mag. of trends in streamflow variability periodicity which focused on four categories: (1) climatology (2) topography (3) basin morphology (4) human disturbance. In order to describe climatology, seven average variables were used: (1) mean annual precipitation $(P_\textit{mean})$ (2) P std. dev (3) mean annual air temp $(T_\textit{mean})$ (4) T std. dev. (5) mean ann. potential evaportranspiration $(PET_\textit{mean}$ (6) PET std. dev. (7) mean ann. dryness index $(PET/P, DI_\textit{mean})$ Topographic variables: (1) mean elevation $(Ele_\textit{mean})$ (2) Ele std. dev. (3) mean slope $(Slp_\textit{mean})$ (4) Slp std. dev.  Additional variables include: (1) mean up slope accumulation $(UAA_\textit{mean})$ (2) UAA std. dev. (3) total basin area (4) values from falcone's 2010 study which measured human distrubance (Iv) At the CONUS scale, a general pattern of decreasing trends in ann. scale streamflow was observed with decrease outnumbering increase but at the sub-CONUS scale, deviation was observed. (IV) Discussion: Decline in annual variability was observed especially in mountainous regions but areas like SE Coastal Plain, Centeral Plains, and Western Xeric ecoregions disagreed. Atmospheric scale process are potentiallu drivers. But overall, results suggest that human activities may magnify or amplify the expression of changes. $P_\textit{mean} \;\&\; DI_\textit{mean}$ were the important variables. Watershed climatology was too varied due to variations amongst the regions being studied 

        \begin{enumerate}
            \item Search engine used: \url{https://search.library.pdx.edu/primo-explore/}
            \item Keywords: "streamflow variability"AND"human activity"
            \item How many results: 8 
            \item How many were peer-reviewed: 8
            \item How many are from 2017 or newer: 5
            \item Calculate \% new peer-reviewed $\approx 62.50$  \%
            \item Link: \url{https://www-sciencedirect-com.proxy.lib.pdx.edu/science/article/pii/S002216941630436X}
            \item type: Peer-reviewed article
        \end{enumerate}

    %2
    \item Fisichelli - \textit{Multiple methods for multiple futures: Integrating qualitative scenario planning and quantitative simulation modeling for natural resource decision making}
        \begin{enumerate}
            \item Search engine used: None 
            \item Keywords: 
            \item How many results:  
            \item How many were peer-reviewed: 
            \item How many are from 2017 or newer: 
            \item Calculate \% new peer-reviewed 
            \item Link: \url{https://link-springer-com.proxy.lib.pdx.edu/content/pdf/10.1007/s12665-018-7305-x.pdf}
            \item type: Peer-reviewed article
        \end{enumerate}

        
    %3
    \item USGS - \textit{Data from simulations of ecological and hydrologic response to climate change scenarios at Wind Cave National Park, South Dakota, 1901-2050}
        \begin{enumerate}
            \item Search engine used: \url{https://catalog.data.gov/}
            \item Keywords: streamflow 
            \item How many results: 662 datasets 
            \item How many were peer-reviewed: 
            \item How many are from 2017 or newer: 
            \item Calculate \% new peer-reviewed 
            \item Link: \url{https://www.sciencebase.gov/catalog/item/get/5a281bbfe4b03852bafe1002}
            \item type: csv file 
        \end{enumerate}


    %4
    \item USGS - \textit{Data from simulations of ecological and hydrologic response to climate change scenarios at Wind Cave National Park, South Dakota, 1901-2050}
        \begin{enumerate}
            \item Search engine used: \url{https://google.com in news section}
            \item Keywords: streamflow and South Dakota 
            \item How many results: 627 
            \item How many were peer-reviewed: 
            \item How many are from 2017 or newer: 
            \item Calculate \% new peer-reviewed 
            \item Link: \url{https://www.hpj.com/ag/_news/dryness-expands-across-high-plains-region-as-planting-begins/article_4c598074-8ae9-11ea-9737-9fc1d3dc9019.html}
            \item type: article 
        \end{enumerate}


    %5
    \item Water data South Dakota
        \begin{enumerate}
            \item Link: \url{https://waterdata.usgs.gov/sd/nwis/current/?type=flow}
        \end{enumerate}

    % 6
    \item Falcone - \textit{Quantifying Human disturbance}\\\\Notes: (I) Abstract: Being able to quantify human disturbances in order to study streamflow is crucial. This study used data from the Geographic Information System (GIS). Datasets of watersheds consisted of 770 of them in the western United State for which the severity of disturbance had previously been classified by the  United States Environmental Protection Agency (USEPA). The final index was validated 192 control watershed sample and about two-thirds (68\%) were correctly classified from least to most-disturbed classification. (II) Introduction: Stream ecosystems are profoundly influenced by human activities. Disturbances include point-source pollution, conversion of natural vegetation to developed land, nutrient and pesticide input from agricultural and urban sources, mining and mineral extraction operation, channel modification, and water impoundment. The primary method for quantifying humnan disturbance is to use a "disturbance index" based GIS data which provides an objective, \textit{a priori} approach to understanding athropengic factor over large areas in a way that would be difficult to achieve with filed studies. In addition we also evaluated the influence of various methods on the performance of the indicies: (1) the use of correlated versus uncorrelated variables, (2) index scoring and weighting methods, (3) assessing which variables most strongly control the index, and (4) determining whether a multimetric index is a more powerful tool than any individual variable. (II) Method: Watersheds ranged in size from 0.6 to 35,110 km squared (median 44 km squared). ites classified as ‘‘reference’’ (hereafter REF) were considered to be representative of near natural or least disturbed conditions in their respective ecoregions. Conversely, sites classified as ‘‘most disturbed’’ (hereafter DIS) were considered to be representative of the most altered or modified by human activities. A third class, ‘‘intermediate’’ (hereafter MED), repre- sented watersheds that have been altered to some intermediate degree. We randomly selected 770 (80\%) of the 962 watersheds for model calibration and reserved 192 (20\%) to independently validate the accuracy of the final best index method. *Refer to table 1*This final selection resulted in a group of six variables: HUDEN, ROADDEN, PESTIC, URBCP_MAINS, DIST_CANAL_NEAR and DAMSTOR. (III) Discussion: The Reduced-Original data also performed better in every case than indices using all original 33 variables; and in nearly every case the Reduced-Synthetic dataset (indices K–T) outperformed both the Redundant dataset and the original 33 variables as well. This strongly suggested that removing redundant or highly correlated variables from the index calcula- tion was beneficial. One aspect of index creation that became evident as the indices were tested was the difference between index scores by region (Fig. 4). By any index method tested, sites in the Plains region clearly had the highest amount of anthropogenic disturbance, followed by the Xeric then Mountains regions. Although limited by the coarse resolution of national-scale GIS coverages, our index correctly classified more than 2/3 of an independent set of sites as either least- or most-disturbed. It is clear that the success of the disturbance index will be dependent on the accuracy and resolution of the GIS data themselves. Nonetheless, as was shown here, thoughtful implementation of the method for creating an index can make a difference in results, even at a broad, regional scale.
        \begin{enumerate}
            \item none 
            \item Keywords: n/a 
            \item How many results:  
            \item How many were peer-reviewed: 
            \item How many are from 2017 or newer: 
            \item Calculate \% new peer-reviewed 
            \item Link: 
            \item type: article 
        \end{enumerate}
    
    

\end{enumerate}

\end{document}
